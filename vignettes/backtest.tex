\documentclass[a4paper]{report}
\usepackage[round]{natbib}

\usepackage{Rnews}
\usepackage{fancyvrb}
\usepackage{Sweave}  

\DefineVerbatimEnvironment{Sinput}{Verbatim}{fontsize=\small,fontshape=sl}
\DefineVerbatimEnvironment{Soutput}{Verbatim}{fontsize=\small}
\DefineVerbatimEnvironment{Scode}{Verbatim}{fontsize=\small,fontshape=sl}

%% \SweaveOpts{prefix.string=graphics/portfolio}

\bibliographystyle{abbrvnat}

\begin{document}
\input{backtest-concordance}
\begin{article}
\title{Backtests}
\author{Kyle Campbell, Jeff Enos, Daniel Gerlanc and David Kane}

%%\VignetteIndexEntry{Using the backtest package}
%%\VignetteDepends{backtest}


\maketitle

\setkeys{Gin}{width=0.95\textwidth}

\section*{Introduction}

The \pkg{backtest} package provides facilities for exploring
portfolio-based conjectures about financial instruments (stocks, bonds,
swaps, options, et cetera).  For example, consider a claim that stocks
for which analysts are raising their earnings estimates perform better
than stocks for which analysts are lowering estimates.  We want to
examine if, on average, stocks with raised estimates have higher
future returns than stocks with lowered estimates and whether this is
true over various time horizons and across different categories of
stocks.  Colloquially, ``backtest'' is the term used in finance for
such tests.

\section*{Background}
To demonstrate the capabilities of the \pkg{backtest} package we will
consider a series of examples based on a single real-world data set.
StarMine\footnote{See www.starmine.com for details.} is a San
Fransisco research company which creates quantitative equity models
for stock selection.  According to the company:

% Note that we make the font size for this quote small and then go
% back to normal.

\small

\begin{quote}
  StarMine Indicator is a 1-100 percentile ranking of stocks that is
  predictive of future analyst revisions. StarMine Indicator improves
  upon basic earnings revisions models by:

\begin{itemize}
\item Explicitly considering management guidance.
  
\item Incorporating SmartEstimates, StarMine's superior estimates
  constructed by putting more weight on the most accurate analysts.
  
\item Using a longer-term (forward 12-month) forecast horizon (in
  addition to the current quarter).

\end{itemize}

StarMine Indicator is positively correlated to future stock price
movements. Top-decile stocks have annually outperformed bottom-decile
stocks by 27 percentage points over the past ten years across all
global regions.
\end{quote}

\normalsize

These ranks and other attributes of stocks are in the
\texttt{starmine} data frame, available as part of the
\pkg{backtest} package.


\begin{Schunk}
\begin{Sinput}
> data(starmine)
> names(starmine)
\end{Sinput}
\begin{Soutput}
 [1] "date"                  
 [2] "id"                    
 [3] "symbol"                
 [4] "name"                  
 [5] "country"               
 [6] "sector"                
 [7] "sec"                   
 [8] "ind"                   
 [9] "size"                  
[10] "smi"                   
[11] "liq"                   
[12] "ret.0.1.m"             
[13] "ret.0.6.m"             
[14] "ret.1.0.m"             
[15] "ret.6.0.m"             
[16] "ret.12.0.m"            
[17] "mn.dollar.volume.20.d" 
[18] "md.dollar.volume.120.d"
[19] "cap.usd"               
[20] "cap"                   
[21] "sales"                 
[22] "net.income"            
[23] "common.equity"         
\end{Soutput}
\end{Schunk}

\texttt{starmine} contains selected attributes such as sector, market
capitalisation, country, and various measures of return for a universe
of approximately 6,000 securities.  The data is on a monthly frequency
from January, 1995 through November, 1995.  The number of observations
varies over time from a low of 4,528 in February to a high of 5,194 in
November.

\begin{Schunk}
\begin{Soutput}
      date count
1995-01-31  4593
1995-02-28  4528
1995-03-31  4569
1995-04-30  4708
1995-05-31  4724
1995-06-30  4748
1995-07-31  4878
1995-08-31  5092
1995-09-30  5185
1995-10-31  5109
1995-11-30  5194
\end{Soutput}
\end{Schunk}

The \texttt{smi} column contains the StarMine Indicator score for each
security and date if available.  Here is a sample of rows
and columns from the data frame:


\begin{Schunk}
\begin{Soutput}
      date          name ret.0.1.m ret.0.6.m smi
1995-01-31   Lojack Corp      0.09       0.8  96
1995-02-28  Raymond Corp      0.05       0.1  85
1995-02-28   Lojack Corp      0.08       0.7  90
1995-03-31   Lojack Corp      0.15       1.0  49
1995-08-31 Supercuts Inc     -0.11      -0.5  57
1995-10-31   Lojack Corp     -0.40      -0.2  22
1995-11-30   Lojack Corp      0.20       0.4  51
\end{Soutput}
\end{Schunk}


Most securities (like LoJack above) have multiple entries in the data
frame, each for a different date.  The row for Supercuts indicates
that, as of the close of business on August 31, 1995, its \texttt{smi}
was 57.  During the month of September, its return (i.e.,
\texttt{ret.0.1.m}) was -11\%.

\section*{A simple backtest}

Backtests are run by calling the function \texttt{backtest} to
produce an object of class \texttt{backtest}.

\begin{Schunk}
\begin{Sinput}
> bt <- backtest(starmine, in.var = "smi", ret.var = "ret.0.1.m", by.period = FALSE)